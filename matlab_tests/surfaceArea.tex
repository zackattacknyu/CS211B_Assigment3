\documentclass[11pt,psfig]{article}
\usepackage{epsfig}
\usepackage{times}
\usepackage{amssymb}
\usepackage{float}

\newcount\refno\refno=1
\def\ref{\the\refno \global\advance\refno by 1}
\def\ux{\underline{x}}
\def\uw{\underline{w}}
\def\bw{\underline{w}}
\def\ut{\underline{\theta}}
\def\umu{\underline{\mu}} 
\def\bmu{\underline{\mu}} 
\def\be{p_e^*}
\newcount\eqnumber\eqnumber=1
\def\eq{\the \eqnumber \global\advance\eqnumber by 1}
\def\eqs{\eq}
\def\eqn{\eqno(\eq)}

 \pagestyle{empty}
\def\baselinestretch{1.1}
\topmargin1in \headsep0.3in
\topmargin0in \oddsidemargin0in \textwidth6.5in \textheight8.5in
\begin{document}
\setlength{\parskip}{1.2ex plus0.3ex minus 0.3ex}

\title{Surface Area for Final Project}
\maketitle
The following page has a good explanation of solid angle as well as the surface area of the patch:
http://mathworld.wolfram.com/SolidAngle.html

From that page we know that
\[
\Omega = \int{ \int{ sin(\phi) \,d\theta \,d\phi }}
\]

We want to find the surface area of a patch $[\phi_1,\phi_2] \times [\theta_1,\theta_2]$. This ends up being
\[
\Omega = \int_{\phi_1}^{\phi_2}{ \int_{\theta_1}^{\theta_2}{ sin(\phi) \,d\theta \,d\phi }}
\]
\[
\Omega = (\theta_2 - \theta_1)\int_{\phi_1}^{\phi_2}{  sin(\phi) \,d\phi }
\]
\[
\Omega = (\theta_1 - \theta_2)(cos\phi_2 - cos\phi_1)
\]
Our current patches are in the form $[x_1,y_1] \times [x_2,y_2]$ which are squares in the unit disk. This page lists more information:
http://luki.webzdarma.cz/eng\_12\_en.htm

\end{document}








